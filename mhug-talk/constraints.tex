% Note on compiling: requires external programs dot (from the graphvix package
% on Debian/Ubuntu) and dot2tex.
% You need to pass the --shell-escape option to pdflatex
% (or use the Makefile, which required latexmk).
%\documentclass[aspectratio=169,handout]{beamer}
%\documentclass[aspectratio=169,handout,hyphens]{beamer} % hyphens option for url package which beamer loads.
\documentclass[aspectratio=169,hyphens]{beamer} % hyphens option for url package which beamer loads.

\def\UrlFont{\small\tt}

\usepackage{minted}
\usepackage{color}
%\usepackage[pgf,dot]{dot2texi}
%\usepackage{tikz}
%\usetikzlibrary{shapes,arrows}

\usetheme{Pittsburgh}
\usecolortheme{beaver}
% \useoutertheme{infolines}

\title{Constraint Programming in Haskell}
\subtitle{Melbourne Haskell Users Group}
\author{David Overton}
\date{29 October 2015}

\AtBeginSection[]
{
	\begin{frame}
		\frametitle{Table of Contents}
		\tableofcontents[currentsection]
	\end{frame}
}

%\lstnewenvironment{code}{\lstset{language=Haskell,basicstyle=\small}}{}

\newminted[code]{haskell}{fontsize=\small}
\newmint[hask]{haskell}{fontsize=\small}
\newminted{prolog}{fontsize=\small}

\definecolor{mygreen}{rgb}{0,0.6,0}
\definecolor{mygrey}{rgb}{0.5,0.5,0.5}
\definecolor{mymauve}{rgb}{0.58,0,0.82}

\newtheorem{observation}[theorem]{Observation}

\newcommand\myheading[1]{%
  \par\bigskip
  {\large\color{blue}#1}\par\smallskip}
\begin{document}

\frame{\titlepage}

\section{Constraint programming}

\begin{frame}
    \frametitle{Constraint programming}
    Constraint programming is a declarative programming paradigm for solving constraint satisfaction problems.
    \pause
    \begin{itemize}
        \item A set of \emph{constraint variables} over a \emph{domain}, e.g. Booleans, integers, reals, finite domain.
        \item A set of \emph{constraints} between those variables.
        \item A \emph{solver} to find solutions to the constraints, i.e. assignments of variables to values in the domain such that all constraints are satisfied.
    \end{itemize}
    \pause
    Applications: planning, scheduling, resource allocation, computer graphics, digital circuit design, programming language analysis, \ldots
\end{frame}

\section{Constraint logic programming}

\begin{frame}
    \frametitle{Constraint logic programming}
    \begin{itemize}
        \item Constraint programming and logic programming work well together.
        \item Many Prolog implementations have built in constraint solvers.
        \item Basic idea:
            \begin{itemize}
                \item add constraints to the constraint store
                \item constraint solver works behind the scenes to propagate constraints
                \item Prolog's backtracking search mechanism to generate solutions
            \end{itemize}
        \item Advantages over pure logic programming:
            \begin{itemize}
                \item ``constrain-and-generate'' rather than ``generate-and-test''
                \item constraint solver can greatly reduce the search space required compared to Prolog's built-in depth-first-search
                \item much more powerful than relying on just unification and backtracking
            \end{itemize}
    \end{itemize}
\end{frame}

\section{Finite domain constraints}

\begin{frame}[fragile]
    \frametitle{Finite domain constraints}

\begin{prologcode}
n_queens(N, Qs) :-
        length(Qs, N),
        Qs ins 1..N,
        safe_queens(Qs).

safe_queens([]).
safe_queens([Q|Qs]) :- safe_queens(Qs, Q, 1), safe_queens(Qs).

safe_queens([], _, _).
safe_queens([Q|Qs], Q0, D0) :-
        Q0 #\= Q,
        abs(Q0 - Q) #\= D0,
        D1 #= D0 + 1,
        safe_queens(Qs, Q0, D1).
\end{prologcode}

\end{frame}

\section{Constraint programming in Haskell}

\begin{frame}[fragile]
    \frametitle{Constraint programming in Haskell}
How can we do something similar in Haskell?\pause{}
\textbf{Use a monad!}
\pause
\begin{itemize}
    \item List monad provides backtracking / search / multiple solutions.
        \pause
    \item Wrap it in a state monad transformer to keep track of the constraint store.
        \pause
\end{itemize}
\begin{code}
-- The FD monad
newtype FD s a = FD { unwrapFD :: StateT (FDState s) [] a }
    deriving (Monad, MonadPlus, MonadState (FDState s))
\end{code}
\pause
\begin{code}
-- Convenient alias for constraint type
type FDConstraint s = FD s ()
\end{code}
\pause
\begin{code}
-- Run the FD monad and produce a lazy list of possible solutions.
runFD :: (forall s . FD s a) -> [a]
runFD fd = evalStateT (unwrapFD fd) initState
\end{code}
\end{frame}

\begin{frame}[fragile]
\begin{code}
-- FD variables
newtype FDVar s = FDVar { unFDVar :: Int } deriving (Ord, Eq)

type VarSupply s = FDVar s

data Domain
    = Set IntSet
    | Range Int Int

data VarInfo s = VarInfo
     { delayedConstraints :: FD s (), domain :: Domain }

type VarMap s = Map (FDVar s) (VarInfo s)

data FDState s = FDState
     { varSupply :: VarSupply s, varMap :: VarMap s }

initState :: FDState s
initState = FDState { varSupply = FDVar 0, varMap = Map.empty }
\end{code}
\end{frame}

\begin{frame}[fragile]
\begin{code}
nQueens :: Int -> FD s [FDVar s]
nQueens n = do
    qs <- newVars n (1, n)
    safeQueens qs
    return qs

safeQueens :: [FDVar s] -> FDConstraint s
safeQueens [] = return ()
safeQueens (q : qs) = safeQueen qs q 1 >> safeQueens qs

safeQueen :: [FDVar s] -> FDVar s -> Int -> FDConstraint s
safeQueen [] _ _ = return ()
safeQueen (q : qs) q0 d = do
   q0 ./=. q 
   q ./=. q0 .+. d
   q ./=. q0 .-. d
   safeQueen qs q0 (d + 1)
\end{code}
\end{frame}

\end{document}
